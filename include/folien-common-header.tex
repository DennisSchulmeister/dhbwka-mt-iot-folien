\beamertemplatenavigationsymbolsempty
\setbeamertemplate{section in toc}[sections numbered]
\setbeamertemplate{subsection in toc}[subsections numbered]
\setbeamertemplate{subsubsection in toc}[subsubsections numbered]
\setbeamerfont{section in toc}{size=\bfseries}
\setbeamerfont{subsection in toc}{size=\footnotesize}
\setbeamerfont{subsubsection in toc}{size=\tiny}
\usepackage{helvet}
\hypersetup{pdfpagelayout={SinglePage},pdfpagetransition={Dissolve}}

\usepackage{calc}
\usepackage{listings}
\usepackage[T1]{fontenc}
\usepackage[utf8]{inputenc}
\usepackage{MnSymbol,wasysym}
\usepackage[ngerman]{babel} \usepackage{csquotes} \MakeOuterQuote{"}
\usepackage{xspace} \newcommand{\latex}{\LaTeX\xspace} \newcommand{\tex}{\TeX\xspace}
\usepackage{tabularx}
\usepackage{tikz}
\usepackage{graphicx}
\usepackage{subfig}
\usepackage{siunitx}
\usepackage{amssymb}
\usepackage{fancyvrb}
\usepackage{array}
\usepackage{courier}
\usepackage{ragged2e}
\usepackage{changepage}

\usepackage{tabto}
\NumTabs{32}

\newcommand{\quotes}[1]{``#1''}
\newcommand{\Href}[1]{\href{#1}{#1}}

% Kompatiblität mit dem Titelmakros der Aufgabenblätter
\newcommand{\ubInstitute}[1]{}
\newcommand{\ubModule}[1]{}
\newcommand{\ubType}[1]{}

% Makro für Folien vor dem Inhaltsverzeichnis (Vorstellung des Dozenten)
\newcommand{\IntroFrames}[1]{#1}

% Hilfsmakro für den Backslash
\newcommand\Slash{\char`\\}

% Absatz mit Blocksatz
\newcommand{\Justified}[1]{
    \parbox{\linewidth}{#1}
}

% Link auf Onlinesimulator
\newcommand{\CircuitJS}[1]{
    \href{#1}{\beamergotobutton{Online ausprobieren}}
}

\newcommand{\LinkButton}[2]{
    \href{#1}{\beamergotobutton{#2}}
}

% Ein Bild über die gesamte Folie
\newcommand{\FullscreenImage}[1]{
    \begin{tikzpicture}[remember picture,overlay]
        \node[at=(current page.center)] {
            \includegraphics[
                %keepaspectratio,
                width=\paperwidth,
                height=\paperheight
            ]{#1}
        };
    \end{tikzpicture}
}

% Syntax Highlighting
% Vgl. https://en.wikibooks.org/wiki/LaTeX/Source_Code_Listings
\definecolor{mymauve}{rgb}{0.58,0,0.82}

\lstset{
    basicstyle=\scriptsize\ttfamily,
    belowcaptionskip=1\baselineskip,
    breaklines=true,
    tabsize=2,
    captionpos=b,
    extendedchars=true,
    showstringspaces=true,
    showtabs=false,
    numbers=left,
    numbersep=5pt,
    numberstyle=\color[RGB]{200,200,200},
    keywordstyle=\color[RGB]{0,0,127},
    commentstyle=\color{Gray},
    identifierstyle=\color[RGB]{80,0,0},
    stringstyle=\color{orange},
    morekeywords={var},
    literate=
      {á}{{\'a}}1 {é}{{\'e}}1 {í}{{\'i}}1 {ó}{{\'o}}1 {ú}{{\'u}}1
      {Á}{{\'A}}1 {É}{{\'E}}1 {Í}{{\'I}}1 {Ó}{{\'O}}1 {Ú}{{\'U}}1
      {à}{{\`a}}1 {è}{{\`e}}1 {ì}{{\`i}}1 {ò}{{\`o}}1 {ù}{{\`u}}1
      {À}{{\`A}}1 {È}{{\'E}}1 {Ì}{{\`I}}1 {Ò}{{\`O}}1 {Ù}{{\`U}}1
      {ä}{{\"a}}1 {ë}{{\"e}}1 {ï}{{\"i}}1 {ö}{{\"o}}1 {ü}{{\"u}}1
      {Ä}{{\"A}}1 {Ë}{{\"E}}1 {Ï}{{\"I}}1 {Ö}{{\"O}}1 {Ü}{{\"U}}1
      {â}{{\^a}}1 {ê}{{\^e}}1 {î}{{\^i}}1 {ô}{{\^o}}1 {û}{{\^u}}1
      {Â}{{\^A}}1 {Ê}{{\^E}}1 {Î}{{\^I}}1 {Ô}{{\^O}}1 {Û}{{\^U}}1
      {Ã}{{\~A}}1 {ã}{{\~a}}1 {Õ}{{\~O}}1 {õ}{{\~o}}1
      {œ}{{\oe}}1 {Œ}{{\OE}}1 {æ}{{\ae}}1 {Æ}{{\AE}}1 {ß}{{\ss}}1
      {ű}{{\H{u}}}1 {Ű}{{\H{U}}}1 {ő}{{\H{o}}}1 {Ő}{{\H{O}}}1
      {ç}{{\c c}}1 {Ç}{{\c C}}1 {ø}{{\o}}1 {å}{{\r a}}1 {Å}{{\r A}}1
      {€}{{\euro}}1 {£}{{\pounds}}1 {«}{{\guillemotleft}}1
      {»}{{\guillemotright}}1 {ñ}{{\~n}}1 {Ñ}{{\~N}}1 {¿}{{?`}}1
}

% Vgl. https://tex.stackexchange.com/a/266705
\lstdefinelanguage{Config}
{
    morecomment=[s][\color{blue}\bfseries]{[}{]},
    morecomment=[l]{\#},
    morecomment=[l]{;},
    commentstyle=\color{gray}\ttfamily,
    morekeywords={},
    otherkeywords={=,:},
}

% Einzelne Folien aus der Kopfnavigation ausnehmen
% Vgl. https://tex.stackexchange.com/a/317833
\makeatletter
\let\beamer@writeslidentry@miniframeson=\beamer@writeslidentry%
\def\beamer@writeslidentry@miniframesoff{%
  \expandafter\beamer@ifempty\expandafter{\beamer@framestartpage}{}% does not happen normally
  {%else
    % removed \addtocontents commands
    \clearpage\beamer@notesactions%
  }
}
\newcommand*{\miniframeson}{\let\beamer@writeslidentry=\beamer@writeslidentry@miniframeson}
\newcommand*{\miniframesoff}{\let\beamer@writeslidentry=\beamer@writeslidentry@miniframesoff}
\makeatother

% Schaubilder mit TikZ
\usetikzlibrary{arrows.meta, positioning, fit, snakes}
